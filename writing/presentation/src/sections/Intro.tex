\begin{frame}{Goal} % Pingu
  Create a tool that propose conjectures for a set of existing definitions in Lean.

  \vspace{1cm}

  A focus on transpilation
\end{frame}

% This project, that we call Pisa, set out to try and accomplish a similar goal to that Hipster.
% That is to say, introduce a conjecture generator to Lean based on user provided types and definitions.
% To accomplish this, tooling of Haskell was used, since they had shown their potential with Hipster, and have since gotten more involved in the form of RoughSpec.
% To utilize the tooling from Haskell, a source to source compilation of some form had to be done. So this is the part which we spent most of our time on.

\begin{frame}[fragile]{Usage} % Pingu
  A code action has been created that has the following syntax
  \vspace{1cm}
  \begin{LeanCode}
    #pisa num? ident+
  \end{LeanCode}
\end{frame}

% To begin, lets see how the tool is actually used by people.
% We can see in the slide the code action that was introduced into Lean.
% It shows how it can be used, and we will go through the details later.

% SWITCH TO EMACS BUFFER HERE
% Maybe write binary tree quickly and show new types quickly
