\section{Enumerated types}\label{sec:result:enum}
The boolean definition shows one of the base data types of which are enumerable.
The definitions used, the generated code, and the generated conjectures in Lean can be found in \cref{app:enum}.
A noteworthy point is that the definitions were written with different types of syntax to show that a user can write in different ways without impacting the semantics.
Using Pisa, the conjectures shown in \cref{eqs:enum} are generated.
In the resulting conjectures one can find conjectures regarding inverse function (\cref{notinverse}), commutativity (\cref{orcom,andcom}), associativity (\cref{orassoc,andassoc}), De Morgan laws (\cref{demorgan1,demorgan2}), among others.
These conjectures portray a baseline for the domain, allowing more intricate proofs to use them.

To measure the relevance of the generated conjectures the following sources of mathlib were used: ``SimpLemmas'' \autocite{SimpLemmas},  ``Bool.and'' \autocite{Bool.And}, ``Bool.or'' \autocite{Bool.Or}, and ``Bool.distrubitivity'' \autocite{Bool.Dist}.
Some of the definitions had to be cut, due to utilizing functions not implemented, or equivalences which is not supported.
The lemmas that is tested against can be seen in \cref{eqs:mathlib_enum}.
This results in a $TP = 22$, $FP = 3$, $FN = 13$, which means that $Recall = 0.6286$ and $Precision = 0.88$.
For the exact list of categorization to generate the values, see \cref{app:overlapenum}.

\newpage

\begin{conjectureset}[H]
\begin{align}
§not§\ §false§ &= §true§ \\
§not§\ §true§ &= §false§ \\
\forall x\ y : \mathbb{B}.\ §and§\ x\ y &= §and§\ y\ x \label{andcom}\\
\forall x : \mathbb{B}.\ §and§\ x\ x &= x \\
\forall x\ y : \mathbb{B}.\ §or§\ x\ y &= §or§\ y\ x \label{orcom}\\
\forall x : \mathbb{B}.\ §or§\ x\ x &= x \\
\forall x : \mathbb{B}.\ §and§\ x\ §false§ &= §false§ \\
\forall x : \mathbb{B}.\ §and§\ x\ §true§ &= x \\
\forall x : \mathbb{B}.\ §or§\ x\ §false§ &= x \\
\forall x : \mathbb{B}.\ §or§\ x\ §true§ &= §true§ \\
\forall x : \mathbb{B}.\ §not§\ (§not§\ x) &= x \label{notinverse}\\
\forall x : \mathbb{B}.\ §and§\ x\ (§not§\ x) &= §false§ \\
\forall x : \mathbb{B}.\ §or§\ x\ (§not§\ x) &= §true§ \\
\forall x\ y\ z : \mathbb{B}.\ §and§\ x\ (§and§\ y\ z) &= §and§\ y\ (§and§\ x\ z) \\
\forall x\ y : \mathbb{B}.\ §and§\ x\ (§or§\ x\ y) &= x \\
\forall x\ y : \mathbb{B}.\ §or§\ x\ (§and§\ x\ y) &= x \\
\forall x\ y\ z : \mathbb{B}.\ §or§\ x\ (§or§\ y\ z) &= §or§\ y\ (§or§\ x\ z) \\
\forall x\ y : \mathbb{B}.\ §and§\ (§not§\ x)\ (§not§\ y) &= §not§\ (§or§\ x\ y) \label{demorgan1}\\
\forall x\ y : \mathbb{B}.\ §and§\ (§not§\ x)\ (§or§\ x\ y) &= §and§\ y\ (§not§\ x) \\
\forall x\ y\ z : \mathbb{B}.\ §and§\ (§or§\ x\ y)\ (§or§\ x\ z) &= §or§\ x\ (§and§\ y\ z) \\
\forall x : \mathbb{B}.\ §and§\ §true§\ x &= x \\
\forall x : \mathbb{B}.\ §or§\ §false§\ x &= x \\
\forall x\ y : \mathbb{B}.\ §or§\ (§not§\ x)\ (§not§\ y) &= §not§\ (§and§\ x\ y) \label{demorgan2}\\
\forall x\ y\ z : \mathbb{B}.\ §and§\ (§and§\ x\ y)\ z &= §and§\ x\ (§and§\ y\ z) \label{andassoc}\\
\forall x\ y\ z : \mathbb{B}.\ §or§\ (§or§\ x\ y)\ z &= §or§\ x\ (§or§\ y\ z) \label{orassoc}
\end{align}
\vspace{-0.9cm}
\caption[Generated by Pisa for the domain $\mathbb{B}$.]{
  Generated by Pisa for the domain $\mathbb{B}$.
  The Lean versions of these conjectures can be seen in \cref{lst:conjecture:enum:output}.
}\label{eqs:enum}
\end{conjectureset}

\begin{conjectureset}[H]
\begin{align}
(!§false§) &= §true§\ \\
(!§true§) &= §false§\ \\
\forall x\ y : \mathbb{B}.\ (x\ \&\&\ y) &= (y\ \&\&\ x)\ \\
\forall b : \mathbb{B}.\ (b\ \&\&\ b) &= b\ \\
\forall x\ y : \mathbb{B}.\ (x\ ||\ y) &= (y\ ||\ x)\ \\
\forall b : \mathbb{B}.\ (b\ ||\ b) &= b\ \\
\forall b : \mathbb{B}.\ (b\ \&\&\ §false§) &= §false§\ \\
\forall b : \mathbb{B}.\ (b\ \&\&\ §true§) &= b\ \\
\forall b : \mathbb{B}.\ (b\ ||\ §false§) &= b\ \\
\forall b : \mathbb{B}.\ (b\ ||\ §true§) &= §true§\ \\
\forall b : \mathbb{B}.\ (!!b) &= b\ \\
\forall x : \mathbb{B}.\ (x\ \&\&\ !x) &= §false§\ \\
\forall x : \mathbb{B}.\ (x\ ||\ !x) &= §true§\ \\
\forall x\ y\ z : \mathbb{B}.\ (x\ \&\&\ (y\ \&\&\ z)) &= (y\ \&\&\ (x\ \&\&\ z))\ \\
\forall x\ y\ z : \mathbb{B}.\ (x\ ||\ (y\ ||\ z)) &= (y\ ||\ (x\ ||\ z))\ \\
\forall x\ y : \mathbb{B}.\ (!(x\ ||\ y)) &= (!x\ \&\&\ !y)\ \\
\forall x\ y\ z : \mathbb{B}.\ (x\ ||\ y\ \&\&\ z) &= ((x\ ||\ y)\ \&\&\ (x\ ||\ z))\ \\
\forall b : \mathbb{B}.\ (§true§\ \&\&\ b) &= b\ \\
\forall b : \mathbb{B}.\ (§false§\ ||\ b) &= b\ \\
\forall x\ y : \mathbb{B}.\ (!(x\ \&\&\ y)) &= (!x\ ||\ !y)\ \\
\forall a\ b\ c : \mathbb{B}.\ (a\ \&\&\ b\ \&\&\ c) &= (a\ \&\&\ (b\ \&\&\ c))\ \\
\forall a\ b\ c : \mathbb{B}.\ (a\ ||\ b\ ||\ c) &= (a\ ||\ (b\ ||\ c))\ \\
\forall b : \mathbb{B}.\ (§true§\ ||\ b) &= §true§\ \\
\forall b : \mathbb{B}.\ (§false§\ \&\&\ b) &= §false§\ \\
\forall a\ b : \mathbb{B}.\ (a\ \&\&\ (a\ \&\&\ b)) &= (a\ \&\&\ b)\ \\
\forall a\ b : \mathbb{B}.\ ((a\ \&\&\ b)\ \&\&\ b) &= (a\ \&\&\ b)\ \\
\forall x : \mathbb{B}.\ (!x\ \&\&\ x) &= §false§\ \\
\forall x\ y\ z : \mathbb{B}.\ ((x\ \&\&\ y)\ \&\&\ z) &= ((x\ \&\&\ z)\ \&\&\ y)\ \\
\forall a\ b : \mathbb{B}.\ (a\ ||\ (a\ ||\ b)) &= (a\ ||\ b)\ \\
\forall a\ b : \mathbb{B}.\ ((a\ ||\ b)\ ||\ b) &= (a\ ||\ b)\ \\
\forall x : \mathbb{B}.\ (!x\ ||\ x) &= §true§\ \\
\forall x\ y\ z : \mathbb{B}.\ ((x\ ||\ y)\ ||\ z) &= ((x\ ||\ z)\ ||\ y)\ \\
\forall x\ y\ z : \mathbb{B}.\ (x\ \&\&\ (y\ ||\ z)) &= (x\ \&\&\ y\ ||\ x\ \&\&\ z)\ \\
\forall x\ y\ z : \mathbb{B}.\ ((x\ ||\ y)\ \&\&\ z) &= (x\ \&\&\ z\ ||\ y\ \&\&\ z)\ \\
\forall x\ y\ z : \mathbb{B}.\ (x\ \&\&\ y\ ||\ z) &= ((x\ ||\ z)\ \&\&\ (y\ ||\ z))
\end{align}
\vspace{-0.9cm}
\caption{Mathlib equivalent for the domain $\mathbb{B}$.}\label{eqs:mathlib_enum}
\end{conjectureset}
