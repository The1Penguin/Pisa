% CREATED BY DAVID FRISK, 2016
\chapter{Results}\label{chp:results}
The result of this thesis can be compared to the aim outlined in \cref{sec:introduction:problem-statement}.
With Pisa, a user is able to generate conjectures based on their own definitions, from a subset of Lean.
It was feasible to do a translation from Lean to Haskell via a combination of the IR and Lambda calculus to encapsulate the different aspects of functions contra types.
A combination of QuickSpec and RoughSpec were then able to generate conjectures that then were translated to Lean theorems, left to be proven by the user.

To show that Pisa is able to translate and conjecture about the goals, multiple data types were chosen that encapsulate the different aspects of \cref{goal:enumerated-types,goal:recursive-types,goal:polymorphic-types}.
Booleans were chosen as the enumerable type, due to it being one of the fundamental types used by most programming languages.
For recursive data types, natural numbers were chosen, encoded as Peano numbers.
Natural numbers show a practical example in which recursion is needed to be able to use the basic functions one would expect.
As for Polymorphic data types, lists and binary trees were chosen.
They both use polymorphism, and are both recursive.
Each data type and their associated functions were implemented.
All the chosen data types and associated functions were able to be translated which means that the implemented Haskell interface is able to support these classes of types.
Furthermore, the chosen data types and their associated functions are able to used for conjecture generation using the translation, thus fulfilling \nref{goal:conjecture}.
Pisa was also able to be utilized from within the editor, using a macro, as outlined in \nref{sec:integration-of-subsystem}, fulfilling \cref{goal:integration}.

To show the relevance (see \cref{goal:relevance}) of the generated conjectures, the data types which has an equivalent in Mathlib \autocite{Mathlib4} will be compared using precision and recall analysis \autocite{PrecisionAndRecall}.
The formulas that will be used can be seen in \cref{eqs:precisionandrecall}.
Recall shows how well Pisa generates all of the potential true cases, while precision measures the degree of conjectures that can be found in mathlib based of the generated ones.
Both of these values range between 0 and 1.
Discussion of these results will be covered in \cref{sec:discussion}

\begin{equationset}[H]
\begin{gather}
  TP = \text{No. lemmas both in generated conjectures and in mathlib} \nonumber \\
  FP = \text{No. lemmas in generated conjectures, but not in mathlib} \nonumber \\
  FN = \text{No. lemmas in mathlib, but not in generated conjectures} \nonumber \\
  Precision = \frac{TP}{TP + FP} \nonumber \\
  Recall = \frac{TP}{TP + FN} \nonumber
\end{gather}
\caption{Precision and recall formulas.}\label{eqs:precisionandrecall}
\end{equationset}

\section{Enumerated types}\label{sec:result:enum}
The boolean definition shows one of the base data types of which are enumerable.
The definitions used, the generated code, and the generated conjectures in Lean can be found in \cref{app:enum}.
A noteworthy point is that the definitions were written with different types of syntax to show that a user can write in different ways without impacting the semantics.
Using Pisa, the conjectures shown in \cref{eqs:enum} are generated.
In the resulting conjectures one can find conjectures regarding inverse function (\cref{notinverse}), commutativity (\cref{orcom,andcom}), associativity (\cref{orassoc,andassoc}), De Morgan laws (\cref{demorgan1,demorgan2}), among others.
These conjectures portray a baseline for the domain, allowing more intricate proofs to use them.

To measure the relevance of the generated conjectures the following sources of mathlib were used: ``SimpLemmas'' \autocite{SimpLemmas},  ``Bool.and'' \autocite{Bool.And}, ``Bool.or'' \autocite{Bool.Or}, and ``Bool.distrubitivity'' \autocite{Bool.Dist}.
Some of the definitions had to be cut, due to utilizing functions not implemented, or equivalences which is not supported.
The lemmas that is tested against can be seen in \cref{eqs:mathlib_enum}.
This results in a $TP = 22$, $FP = 3$, $FN = 13$, which means that $Recall = 0.6286$ and $Precision = 0.88$.
For the exact list of categorization to generate the values, see \cref{app:overlapenum}.

\newpage

\begin{conjectureset}[H]
\begin{align}
§not§\ §false§ &= §true§ \\
§not§\ §true§ &= §false§ \\
\forall x\ y : \mathbb{B}.\ §and§\ x\ y &= §and§\ y\ x \label{andcom}\\
\forall x : \mathbb{B}.\ §and§\ x\ x &= x \\
\forall x\ y : \mathbb{B}.\ §or§\ x\ y &= §or§\ y\ x \label{orcom}\\
\forall x : \mathbb{B}.\ §or§\ x\ x &= x \\
\forall x : \mathbb{B}.\ §and§\ x\ §false§ &= §false§ \\
\forall x : \mathbb{B}.\ §and§\ x\ §true§ &= x \\
\forall x : \mathbb{B}.\ §or§\ x\ §false§ &= x \\
\forall x : \mathbb{B}.\ §or§\ x\ §true§ &= §true§ \\
\forall x : \mathbb{B}.\ §not§\ (§not§\ x) &= x \label{notinverse}\\
\forall x : \mathbb{B}.\ §and§\ x\ (§not§\ x) &= §false§ \\
\forall x : \mathbb{B}.\ §or§\ x\ (§not§\ x) &= §true§ \\
\forall x\ y\ z : \mathbb{B}.\ §and§\ x\ (§and§\ y\ z) &= §and§\ y\ (§and§\ x\ z) \\
\forall x\ y : \mathbb{B}.\ §and§\ x\ (§or§\ x\ y) &= x \\
\forall x\ y : \mathbb{B}.\ §or§\ x\ (§and§\ x\ y) &= x \\
\forall x\ y\ z : \mathbb{B}.\ §or§\ x\ (§or§\ y\ z) &= §or§\ y\ (§or§\ x\ z) \\
\forall x\ y : \mathbb{B}.\ §and§\ (§not§\ x)\ (§not§\ y) &= §not§\ (§or§\ x\ y) \label{demorgan1}\\
\forall x\ y : \mathbb{B}.\ §and§\ (§not§\ x)\ (§or§\ x\ y) &= §and§\ y\ (§not§\ x) \\
\forall x\ y\ z : \mathbb{B}.\ §and§\ (§or§\ x\ y)\ (§or§\ x\ z) &= §or§\ x\ (§and§\ y\ z) \\
\forall x : \mathbb{B}.\ §and§\ §true§\ x &= x \\
\forall x : \mathbb{B}.\ §or§\ §false§\ x &= x \\
\forall x\ y : \mathbb{B}.\ §or§\ (§not§\ x)\ (§not§\ y) &= §not§\ (§and§\ x\ y) \label{demorgan2}\\
\forall x\ y\ z : \mathbb{B}.\ §and§\ (§and§\ x\ y)\ z &= §and§\ x\ (§and§\ y\ z) \label{andassoc}\\
\forall x\ y\ z : \mathbb{B}.\ §or§\ (§or§\ x\ y)\ z &= §or§\ x\ (§or§\ y\ z) \label{orassoc}
\end{align}
\vspace{-0.9cm}
\caption[Generated by Pisa for the domain $\mathbb{B}$.]{
  Generated by Pisa for the domain $\mathbb{B}$.
  The Lean versions of these conjectures can be seen in \cref{lst:conjecture:enum:output}.
}\label{eqs:enum}
\end{conjectureset}

\begin{conjectureset}[H]
\begin{align}
(!§false§) &= §true§\ \\
(!§true§) &= §false§\ \\
\forall x\ y : \mathbb{B}.\ (x\ \&\&\ y) &= (y\ \&\&\ x)\ \\
\forall b : \mathbb{B}.\ (b\ \&\&\ b) &= b\ \\
\forall x\ y : \mathbb{B}.\ (x\ ||\ y) &= (y\ ||\ x)\ \\
\forall b : \mathbb{B}.\ (b\ ||\ b) &= b\ \\
\forall b : \mathbb{B}.\ (b\ \&\&\ §false§) &= §false§\ \\
\forall b : \mathbb{B}.\ (b\ \&\&\ §true§) &= b\ \\
\forall b : \mathbb{B}.\ (b\ ||\ §false§) &= b\ \\
\forall b : \mathbb{B}.\ (b\ ||\ §true§) &= §true§\ \\
\forall b : \mathbb{B}.\ (!!b) &= b\ \\
\forall x : \mathbb{B}.\ (x\ \&\&\ !x) &= §false§\ \\
\forall x : \mathbb{B}.\ (x\ ||\ !x) &= §true§\ \\
\forall x\ y\ z : \mathbb{B}.\ (x\ \&\&\ (y\ \&\&\ z)) &= (y\ \&\&\ (x\ \&\&\ z))\ \\
\forall x\ y\ z : \mathbb{B}.\ (x\ ||\ (y\ ||\ z)) &= (y\ ||\ (x\ ||\ z))\ \\
\forall x\ y : \mathbb{B}.\ (!(x\ ||\ y)) &= (!x\ \&\&\ !y)\ \\
\forall x\ y\ z : \mathbb{B}.\ (x\ ||\ y\ \&\&\ z) &= ((x\ ||\ y)\ \&\&\ (x\ ||\ z))\ \\
\forall b : \mathbb{B}.\ (§true§\ \&\&\ b) &= b\ \\
\forall b : \mathbb{B}.\ (§false§\ ||\ b) &= b\ \\
\forall x\ y : \mathbb{B}.\ (!(x\ \&\&\ y)) &= (!x\ ||\ !y)\ \\
\forall a\ b\ c : \mathbb{B}.\ (a\ \&\&\ b\ \&\&\ c) &= (a\ \&\&\ (b\ \&\&\ c))\ \\
\forall a\ b\ c : \mathbb{B}.\ (a\ ||\ b\ ||\ c) &= (a\ ||\ (b\ ||\ c))\ \\
\forall b : \mathbb{B}.\ (§true§\ ||\ b) &= §true§\ \\
\forall b : \mathbb{B}.\ (§false§\ \&\&\ b) &= §false§\ \\
\forall a\ b : \mathbb{B}.\ (a\ \&\&\ (a\ \&\&\ b)) &= (a\ \&\&\ b)\ \\
\forall a\ b : \mathbb{B}.\ ((a\ \&\&\ b)\ \&\&\ b) &= (a\ \&\&\ b)\ \\
\forall x : \mathbb{B}.\ (!x\ \&\&\ x) &= §false§\ \\
\forall x\ y\ z : \mathbb{B}.\ ((x\ \&\&\ y)\ \&\&\ z) &= ((x\ \&\&\ z)\ \&\&\ y)\ \\
\forall a\ b : \mathbb{B}.\ (a\ ||\ (a\ ||\ b)) &= (a\ ||\ b)\ \\
\forall a\ b : \mathbb{B}.\ ((a\ ||\ b)\ ||\ b) &= (a\ ||\ b)\ \\
\forall x : \mathbb{B}.\ (!x\ ||\ x) &= §true§\ \\
\forall x\ y\ z : \mathbb{B}.\ ((x\ ||\ y)\ ||\ z) &= ((x\ ||\ z)\ ||\ y)\ \\
\forall x\ y\ z : \mathbb{B}.\ (x\ \&\&\ (y\ ||\ z)) &= (x\ \&\&\ y\ ||\ x\ \&\&\ z)\ \\
\forall x\ y\ z : \mathbb{B}.\ ((x\ ||\ y)\ \&\&\ z) &= (x\ \&\&\ z\ ||\ y\ \&\&\ z)\ \\
\forall x\ y\ z : \mathbb{B}.\ (x\ \&\&\ y\ ||\ z) &= ((x\ ||\ z)\ \&\&\ (y\ ||\ z))
\end{align}
\vspace{-0.9cm}
\caption{Mathlib equivalent for the domain $\mathbb{B}$.}\label{eqs:mathlib_enum}
\end{conjectureset}

\section{Recursive types}\label{sec:result:recursive}
For recursive data types, the Peano encoding of the natural numbers was chosen.
The definitions used, the generated code, and the generated conjectures in lean can be found in \cref{app:recursive}.
Using Pisa, the conjectures shown in \cref{eqs:recursive} are generated.
In the resulting conjectures one can find conjectures regarding identity elements (\cref{addxzero,addzerox,multone}), zero property (\cref{multzero}), commutativity (\cref{multcom,addcom}), associativity (\cref{addassoc,multassoc}), among others.

To measure the relevance of the generated conjectures the following sources of mathlib were used: ``Nat.add theorems'' \autocite{Nat.add} and ``Nat.mul theorems'' \autocite{Nat.mul}.
Some of the definitions had to be cut, due to utilizing functions not implemented, or equivalences which is not supported.
The lemmas that is tested against can be seen in \cref{eqs:mathlib_recursive}.
This results in a $TP = 12$, $FP = 3$, $FN = 14$, which means that $Recall = 0.4615$ and $Precision = 0.8$.
For the exact list of categorization to generate the values, see \cref{app:overlaprecursive}.

\begin{conjectureset}[H]
\begin{align}
\forall x\ y: \mathbb{N}.\ §add§\ x\ y &= §add§\ y\ x \label{addcom}\\
\forall x\ y: \mathbb{N}.\ §mult§\ x\ y &= §mult§\ y\ x \label{multcom}\\
\forall x: \mathbb{N}.\ §add§\ x\ §zero§ &= x \label{addxzero}\\
\forall x: \mathbb{N}.\ §mult§\ x\ §zero§ &= §zero§ \label{multzero}\\
\forall x\ y: \mathbb{N}.\ §add§\ x\ (§succ§\ y) &= §succ§\ (§add§\ x\ y)\\
\forall x: \mathbb{N}.\ §mult§\ x\ (§succ§\ §zero§) &= x \label{multone}\\
\forall x\ y\ z: \mathbb{N}.\ §add§\ x\ (§add§\ y\ z) &= §add§\ y\ (§add§\ x\ z)\\
\forall x\ y: \mathbb{N}.\ §add§\ x\ (§mult§\ x\ y) &= §mult§\ x\ (§succ§\ y)\\
\forall x\ y: \mathbb{N}.\ §mult§\ x\ (§add§\ y\ y) &= §mult§\ y\ (§add§\ x\ x)\\
\forall x\ y\ z: \mathbb{N}.\ §mult§\ x\ (§mult§\ y\ z) &= §mult§\ y\ (§mult§\ x\ z)\\
\forall x\ y\ z: \mathbb{N}.\ §add§\ (§mult§\ x\ y)\ (§mult§\ x\ z) &= §mult§\ x\ (§add§\ y\ z)\\
\forall x: \mathbb{N}.\ \nonumber \\
  §succ§\ (§mult§\ x\ (§succ§\ (§succ§\ (§succ§\ x)))) &= §add§\ x\ (§mult§\ (§succ§\ x)\ (§succ§\ x))\\
\forall x: \mathbb{N}.\ §add§\ §zero§\ x &= x \label{addzerox}\\
\forall x\ y\ z: \mathbb{N}.\ §add§\ (§add§\ x\ y)\ z &= §add§\ x\ (§add§\ y\ z) \label{addassoc}\\
\forall x\ y\ z: \mathbb{N}.\ §mult§\ (§mult§\ x\ y)\ z &= §mult§\ x\ (§mult§\ y\ z) \label{multassoc}
\end{align}
\vspace{-0.9cm}
\caption[Generated by Pisa for the domain $\mathbb{N}$.]{
  Generated by Pisa for the domain $\mathbb{N}$.
  The Lean versions of these conjectures can be seen in \cref{lst:conjecture:recursive:output}.
}\label{eqs:recursive}
\end{conjectureset}

\begin{conjectureset}[H]
\begin{align}
\forall n\ m : \mathbb{N}.\ n + m &= m + n \\
\forall n\ m : \mathbb{N}.\ n * m &= m * n \\
\forall n : \mathbb{N}.\ n * 0 &= 0 \\
\forall n\ m : \mathbb{N}.\ n + §succ§\ m &= §succ§\ (n + m) \\
\forall n : \mathbb{N}.\ n * 1 &= n \\
\forall n\ m\ k : \mathbb{N}.\ n + (m + k) &= m + (n + k) \\
(n m : Nat) : n * (m + 1) &= n * m + n \\
\forall n\ m\ k : \mathbb{N}.\ n * (m * k) &= m * (n * k) \\
\forall n\ m\ k : \mathbb{N}.\ n * (m + k) &= n * m + n * k \\
\forall n : \mathbb{N}.\ 0 + n &= n \\
\forall n\ m\ k : \mathbb{N}.\ (n + m) + k &= n + (m + k) \\
\forall n\ m\ k : \mathbb{N}.\ (n * m) * k &= n * (m * k) \\
\forall n\ m : \mathbb{N}.\ (§succ§\ n) + m &= §succ§\ (n + m) \\
\forall n : \mathbb{N}.\ n + 1 &= §succ§\ n \\
\forall n : \mathbb{N}.\ §succ§\ n &= n + 1 \\
\forall n\ m\ k : \mathbb{N}.\ (n + m) + k &= (n + k) + m \\
\forall n\ m : \mathbb{N}.\ n * §succ§\ m &= n * m + n \\
\forall n : \mathbb{N}.\ 0 * n &= 0 \\
\forall n\ m : \mathbb{N}.\ (§succ§\ n) * m &= (n * m) + m \\
\forall n\ m : \mathbb{N}.\ (n + 1) * m &= (n * m) + m \\
\forall n : \mathbb{N}.\ 1 * n &= n \\
\forall n\ m\ k : \mathbb{N}.\ (n + m) * k &= n * k + m * k \\
\forall n\ m\ k : \mathbb{N}.\ n * (m + k) &= n * m + n * k \\
\forall n\ m\ k : \mathbb{N}.\ (n + m) * k &= n * k + m * k \\
\forall n : \mathbb{n}.\ n * 2 &= n + n \\
\forall n : \mathbb{n}.\ 2 * n &= n + n
\end{align}
\vspace{-0.9cm}
\caption{Mathlib equivalent for the domain $\mathbb{N}$.}\label{eqs:mathlib_recursive}
\end{conjectureset}

\section{Polymorphic types}\label{sec:result:poly}
For polymorphic data types, lists and binary trees were defined.
The definitions used, the generated code, and the generated conjectures in lean can be found in \cref{app:poly}.
Using Pisa, the conjectures shown in conjecture set~\ref{eqs:list} and~\ref{eqs:tree} are generated. % An ugly hack, didn't compile correctly otherwise
In the resulting conjectures one can find conjectures regarding inverse function (\cref{reverseinverse,swapinverse}), identity elements (\cref{identityElement1,identityElement2}), associativity (\cref{appendassoc}), fixpoints (\cref{fixpoint1,fixpoint2,swapfixpoint,rightmostfixpoint,leftmostfixpoint}), and combinations of functions that are equal to another function (\cref{defeq1,defeq2}).

To measure the relevance of the generated conjectures for §List§, since §Tree§ doesn't have an equivalent implementation, the following sources of mathlib were used: ``List.Lemmas.reverse'' \autocite{Lemmas.reverse}, ``List.append'' \autocite{List.append}, and ``List.reverse'' \autocite{List.reverse}.
Some of the definitions had to be cut, due to utilizing functions not implemented, or equivalences which is not supported.
The lemmas that is tested against can be seen in \cref{eqs:mathlib_recursive}.
This results in a $TP = 7$, $FP = 2$, $FN = 2$, which means that $Recall = 0.7778$ and $Precision = 0.7778$ for lists.
For the exact list of categorization to generate the values, see \cref{app:overlappolymorphic}.

As mentioned, §Tree§ was not implemented in the same way as mathlib \autocite{Tree}.
The implementation that can be found in mathlib has values in nodes instead of leaves, and the proofs associated are mostly about size and map functions.

\vspace{-0.8cm}
\begin{conjectureset}[H]
\begin{align}
§reverse§\ §Nil§ &= §Nil§ \label{fixpoint1}\\
\forall \alpha : §Type§.\ \forall x : §List§\ \alpha.\ §append§\ x\ §Nil§ &= x \label{identityElement1}\\
\forall \alpha : §Type§.\ \forall x : §List§\ \alpha.\ §append§\ §Nil§\ x &= x \label{identityElement2}\\
\forall \alpha : §Type§.\ \forall x : §List§\ \alpha.\ §reverse§\ (§reverse§\ x) &= x \label{reverseinverse}\\
\forall \alpha : §Type§.\ \forall x : \alpha.\ §reverse§\ (§Cons§\ x\ §Nil§) &= §Cons§\ x\ §Nil§ \label{fixpoint2}\\
\forall \alpha : §Type§.\ \forall x\ y\ z : §List§\ \alpha.\ \nonumber\\
  §append§\ (§append§\ x\ y)\ z &= §append§\ x\ (§append§\ y\ z) \label{appendassoc}\\
\forall \alpha : §Type§.\ \forall x : \alpha.\ \forall y\ z : §List§\ \alpha.\ \nonumber\\
  §Cons§\ x\ (§append§\ y\ z) &= §append§\ (§Cons§\ x\ y)\ z\\
\forall \alpha : §Type§.\ \forall x\ y : §List§\ \alpha.\ \nonumber\\
  §append§\ (§reverse§\ x)\ (§reverse§\ y) &= §reverse§\ (§append§\ y\ x)\\
\forall \alpha : §Type§.\ \forall x : §List§\ \alpha.\ \forall y\ z : \alpha.\ \nonumber\\
  §append§\ x\ (§Cons§\ y\ (§Cons§\ z\ §Nil§)) &= \nonumber \\
  \quad §reverse§\ (§Cons§\ z\ (§Cons§\ y\ (§reverse§\ x)))
\end{align}
\vspace{-0.9cm}
\caption[Generated by Pisa for the domain $§List§\ \alpha$.]{
  Generated by Pisa for the domain $§List§\ \alpha$.
  The Lean versions of these conjectures can be seen in \cref{lst:conjecture:polymorphic:output}.
}\label{eqs:list}
\end{conjectureset}

\vspace{-1.1cm}
\begin{conjectureset}[H]
\begin{align}
\forall \alpha : §Type§.\ \forall x : \alpha.\ §leftmost§\ (§Leaf§\ x) &= x \label{leftmostfixpoint}\\
\forall \alpha : §Type§.\ \forall x : §Tree§\ \alpha.\ §leftmost§\ (§swap§\ x) &= §rightmost§\ x \label{defeq1}\\
\forall \alpha : §Type§.\ \forall x : \alpha.\ §rightmost§\ (§Leaf§\ x) &= x \label{rightmostfixpoint}\\
\forall \alpha : §Type§.\ \forall x : §Tree§\ \alpha.\ §rightmost§\ (§swap§\ x) &= §leftmost§\ x \label{defeq2}\\
\forall \alpha : §Type§.\ \forall x : \alpha.\ §swap§\ (§Leaf§\ x) &= §Leaf§\ x \label{swapfixpoint}\\
\forall \alpha : §Type§.\ \forall x : §Tree§\ \alpha.\ §swap§\ (§swap§\ x) &= x \label{swapinverse}\\
\forall \alpha : §Type§.\ \forall x\ y : §Tree§\ \alpha.\ §leftmost§\ (§Node§\ x\ y) &= §leftmost§\ x\\
\forall \alpha : §Type§.\ \forall x\ y : §Tree§\ \alpha.\ §rightmost§\ (§Node§\ x\ y) &= §rightmost§\ y\\
\forall \alpha : §Type§.\ \forall x\ y : §Tree§\ \alpha.\ §Node§\ (§swap§\ x)\ (§swap§\ y) &= §swap§\ (§Node§\ y\ x)
\end{align}
\vspace{-0.9cm}
\caption[Generated by Pisa for the domain $§Tree§\ \alpha$.]{
  Generated by Pisa for the domain $§Tree§\ \alpha$.
  The Lean versions of these conjectures can be seen in \cref{lst:conjecture:polymorphic2:output}.
}\label{eqs:tree}
\end{conjectureset}

\begin{conjectureset}[H]
\begin{align}
§reverse§\ §[]§ &= §[]§ \\
\forall \alpha : §Type§.\ \forall as : §List§\ \alpha.\ as\ §++§\ §[]§ &= as \\
\forall \alpha : §Type§.\ \forall as : §List§\ \alpha.\ §[]§\ §++§\ as &= as \\
\forall \alpha : §Type§.\ \forall as : §List§\ \alpha.\ as.§reverse§.§reverse§ &= as \\
\forall \alpha : §Type§.\ \forall as\ bs\ cs : §List§\ \alpha.\ \nonumber \\
  (as\ §++§\ bs)\ §++§\ cs &= as\ §++§\ (bs\ §++§\ cs) \\
\forall \alpha : §Type§.\ \forall a : \alpha.\ \forall as\ bs : §List§\ \alpha.\ \nonumber \\
  (a::as)\ §++§\ bs &= a::(as\ §++§\ bs) \\
\forall \alpha : §Type§.\ \forall as\ bs : §List§\ \alpha.\ \nonumber \\
  (as\ §++§\ bs).§reverse§ &= bs.§reverse§\ §++§\ as.§reverse§ \\
\forall \alpha : §Type§.\ \forall as : §List§\ \alpha.\ \forall b : \alpha.\ \forall bs : §List§\ \alpha.\ \nonumber \\
  as\ §++§\ b :: bs &= as\ §++§\ §[§b§]§\ §++§\ bs \\
\forall \alpha : §Type§.\ \forall a : \alpha.\ \forall as : §List§\ \alpha.\ \nonumber \\
  §reverse§\ (a :: as) &= §reverse§\ as\ §++§\ §[§a§]§
\end{align}
\vspace{-0.9cm}
\caption{Mathlib equivalent for the domain $§List§\ \alpha$.}\label{eqs:mathlib_polymorphic}
\end{conjectureset}

