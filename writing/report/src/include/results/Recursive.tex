\section{Recursive types}\label{sec:result:recursive}
For recursive data types, the Peano encoding of the natural numbers was chosen.
The definitions used, the generated code, and the generated conjectures in lean can be found in \cref{app:recursive}.
Using Pisa, the conjectures shown in \cref{eqs:recursive} are generated.
In the resulting conjectures one can find conjectures regarding identity elements (\cref{addxzero,addzerox,multone}), zero property (\cref{multzero}), commutativity (\cref{multcom,addcom}), associativity (\cref{addassoc,multassoc}), among others.

To measure the relevance of the generated conjectures the following sources of mathlib were used: ``Nat.add theorems'' \autocite{Nat.add} and ``Nat.mul theorems'' \autocite{Nat.mul}.
Some of the definitions had to be cut, due to utilizing functions not implemented, or equivalences which is not supported.
The lemmas that is tested against can be seen in \cref{eqs:mathlib_recursive}.
This results in a $TP = 12$, $FP = 3$, $FN = 14$, which means that $Recall = 0.4615$ and $Precision = 0.8$.
For the exact list of categorization to generate the values, see \cref{app:overlaprecursive}.

\begin{conjectureset}[H]
\begin{align}
\forall x\ y: \mathbb{N}.\ §add§\ x\ y &= §add§\ y\ x \label{addcom}\\
\forall x\ y: \mathbb{N}.\ §mult§\ x\ y &= §mult§\ y\ x \label{multcom}\\
\forall x: \mathbb{N}.\ §add§\ x\ §zero§ &= x \label{addxzero}\\
\forall x: \mathbb{N}.\ §mult§\ x\ §zero§ &= §zero§ \label{multzero}\\
\forall x\ y: \mathbb{N}.\ §add§\ x\ (§succ§\ y) &= §succ§\ (§add§\ x\ y)\\
\forall x: \mathbb{N}.\ §mult§\ x\ (§succ§\ §zero§) &= x \label{multone}\\
\forall x\ y\ z: \mathbb{N}.\ §add§\ x\ (§add§\ y\ z) &= §add§\ y\ (§add§\ x\ z)\\
\forall x\ y: \mathbb{N}.\ §add§\ x\ (§mult§\ x\ y) &= §mult§\ x\ (§succ§\ y)\\
\forall x\ y: \mathbb{N}.\ §mult§\ x\ (§add§\ y\ y) &= §mult§\ y\ (§add§\ x\ x)\\
\forall x\ y\ z: \mathbb{N}.\ §mult§\ x\ (§mult§\ y\ z) &= §mult§\ y\ (§mult§\ x\ z)\\
\forall x\ y\ z: \mathbb{N}.\ §add§\ (§mult§\ x\ y)\ (§mult§\ x\ z) &= §mult§\ x\ (§add§\ y\ z)\\
\forall x: \mathbb{N}.\ \nonumber \\
  §succ§\ (§mult§\ x\ (§succ§\ (§succ§\ (§succ§\ x)))) &= §add§\ x\ (§mult§\ (§succ§\ x)\ (§succ§\ x))\\
\forall x: \mathbb{N}.\ §add§\ §zero§\ x &= x \label{addzerox}\\
\forall x\ y\ z: \mathbb{N}.\ §add§\ (§add§\ x\ y)\ z &= §add§\ x\ (§add§\ y\ z) \label{addassoc}\\
\forall x\ y\ z: \mathbb{N}.\ §mult§\ (§mult§\ x\ y)\ z &= §mult§\ x\ (§mult§\ y\ z) \label{multassoc}
\end{align}
\vspace{-0.9cm}
\caption[Generated by Pisa for the domain $\mathbb{N}$.]{
  Generated by Pisa for the domain $\mathbb{N}$.
  The Lean versions of these conjectures can be seen in \cref{lst:conjecture:recursive:output}.
}\label{eqs:recursive}
\end{conjectureset}

\begin{conjectureset}[H]
\begin{align}
\forall n\ m : \mathbb{N}.\ n + m &= m + n \\
\forall n\ m : \mathbb{N}.\ n * m &= m * n \\
\forall n : \mathbb{N}.\ n * 0 &= 0 \\
\forall n\ m : \mathbb{N}.\ n + §succ§\ m &= §succ§\ (n + m) \\
\forall n : \mathbb{N}.\ n * 1 &= n \\
\forall n\ m\ k : \mathbb{N}.\ n + (m + k) &= m + (n + k) \\
(n m : Nat) : n * (m + 1) &= n * m + n \\
\forall n\ m\ k : \mathbb{N}.\ n * (m * k) &= m * (n * k) \\
\forall n\ m\ k : \mathbb{N}.\ n * (m + k) &= n * m + n * k \\
\forall n : \mathbb{N}.\ 0 + n &= n \\
\forall n\ m\ k : \mathbb{N}.\ (n + m) + k &= n + (m + k) \\
\forall n\ m\ k : \mathbb{N}.\ (n * m) * k &= n * (m * k) \\
\forall n\ m : \mathbb{N}.\ (§succ§\ n) + m &= §succ§\ (n + m) \\
\forall n : \mathbb{N}.\ n + 1 &= §succ§\ n \\
\forall n : \mathbb{N}.\ §succ§\ n &= n + 1 \\
\forall n\ m\ k : \mathbb{N}.\ (n + m) + k &= (n + k) + m \\
\forall n\ m : \mathbb{N}.\ n * §succ§\ m &= n * m + n \\
\forall n : \mathbb{N}.\ 0 * n &= 0 \\
\forall n\ m : \mathbb{N}.\ (§succ§\ n) * m &= (n * m) + m \\
\forall n\ m : \mathbb{N}.\ (n + 1) * m &= (n * m) + m \\
\forall n : \mathbb{N}.\ 1 * n &= n \\
\forall n\ m\ k : \mathbb{N}.\ (n + m) * k &= n * k + m * k \\
\forall n\ m\ k : \mathbb{N}.\ n * (m + k) &= n * m + n * k \\
\forall n\ m\ k : \mathbb{N}.\ (n + m) * k &= n * k + m * k \\
\forall n : \mathbb{n}.\ n * 2 &= n + n \\
\forall n : \mathbb{n}.\ 2 * n &= n + n
\end{align}
\vspace{-0.9cm}
\caption{Mathlib equivalent for the domain $\mathbb{N}$.}\label{eqs:mathlib_recursive}
\end{conjectureset}
