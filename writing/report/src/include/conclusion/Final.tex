\section{Final remarks}\label{sec:final-remarks}
Pisa is able to generate conjectures for Lean and translate a subset of Lean code into Haskell.
The subset includes data types that are either enumerable, recursive or polymorphic.
With a translation, Pisa is able to generate conjectures using QuickSpec and RoughSpec, which then can be ported back into Lean for the user to prove.
Through precision and recall analysis, it was shown that Pisa has high precision ($80\%-90\%$), but a wider range of recall ($45\%-80\%$).

Pisa was integrated into a macro with an associated code action.
This allows users to utilize Pisa within their editor while adding or changing definitions.

There are some limitations, mostly regarding more complex types.
For example, instantiating polymorphic types (such as lists of natural numbers) is not supported.
Further, dependent types and type classes are not constructable, which limits the potential of Pisa.

An extension to Pisa would be that of dependent types and type classes.
To attempt this, our recommendation would be to reimplement the full functionality in Lean.
We believe this has the best potential to be successful.
Implementing Pisa fully in Lean would also allow for the usage of instantiation of Polymorphic data types.
