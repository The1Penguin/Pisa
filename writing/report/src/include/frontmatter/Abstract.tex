% CREATED BY DAVID FRISK, 2016
\oneLineTitle\\
\oneLineSubtitle\\
Nor Führ\\
Erik Nygren\\
Department of Computer Science and Engineering\\
Chalmers University of Technology and University of Gothenburg

\thispagestyle{plain} % Suppress header
\pdfbookmark{Abstract}{abstract}
\section*{Abstract}
As time has passed, the rigour needed for mathematical proofs has increased.
Interactive theorem provers were introduced to help with proof verification during proof development.
This project aims to aid users while they use a particular interactive theorem prover, known as Lean 4, by facilitating the use of conjecture generation for the language.
Conjecture generation can help users understand the domain that they are working in, or automate what proofs might be needed for a more intricate proof.

To facilitate conjecture generation in Lean, a tool called Pisa was introduced.
It utilizes tooling from Haskell to generate conjectures.
To be able to utilize Haskell tooling for conjecture generation, a translation from Lean to Haskell was created, that supports data type translation and has an interpreter for functions.
The translation works for enumerable, recursive, and polymorphic data types.
With the translation, generation of conjectures is performed and then translated back into Lean code for the user to prove them.
This was all tied into a macro, that is callable from the users' editor, meaning that usage of Pisa can be done while adding and modifying definitions.


% KEYWORDS (MAXIMUM 10 WORDS)
\vfill
Keywords: Transpilation, Conjecture generation, Theory exploration, Interactive theorem prover, Interpreter.

\cleardoublepage
