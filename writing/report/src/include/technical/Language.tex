\section{The supported subset of Lean}\label{sec:supported-subset-of-lean}
Pisa tries to support most of Lean.
However, there are some features that are not able to be translated with the current implementation.

Enumerable and recursive data types are supported and functions utilizing them are able to be represented.
Polymorphic data types are also representable, however, instantiating them with a type is not.
This comes from the conversion to monomorphic representations, outlined in \cref{sec:pisa-haskell:translation}.

Dependent types are not supported.
The construction of values for these types hinders the implementation since the IR is able to handle certain aspects of dependent types, such as type level values in a constructor.
But the types for functions would not be able to be converted as that would require dependent types in Haskell.

Type classes is another aspect that could be difficult to represent.
Depending on what the type class encodes, it could be doable, but that would require knowledge of what the intention about said type class is aiming to do.
To implement type classes, one would either have to translate the definitions of the class, or implement specific versions of the common ones.
Neither of these approaches were taken for this thesis.
Further, a translation of the classes would also have to be supplied to the IR.

Not supporting type classes means that §IO§ is not representable.
Further, §IO§ utilizes a lot of opaque definitions.
Opaque definitions are definitions that are not exposed to the kernel~\autocite{Definitions}.
For §IO§ specifically, these functions are usually implemented by the runtime.

Higher order functions are not able to be represented using the current implementation.
By default, QuickCheck is able to generate functions for usage, which enables QuickSpec and RoughSpec to use higher order functions.
With the current implementation of Pisa, this would require a layer between the generation of these functions, and their usage, in order to translate them into Lean IR.
