\section{Partial and unsafe definitions}\label{sec:partial-and-unsafe-definitons}
One difference between Lean and Haskell is how a function is defined.
If a function doesn't properly define all the cases for a match in Lean it will not type check and therefore, not be able to be translated.
In Haskell, this is a warning and will give an error at runtime.
This means that by translating functions from Lean to Haskell, functions which do not match all the cases cannot occur.
However, functions using the \lean{sorry} keyword are still translated.
The output of these functions utilize a lot of built-in definitions (handled by the runtime) that are not added by the exporter.
These built-in definitions are not implemented by the interpreter.
This means that these functions will potentially fail at runtime, which is to be expected.
QuickSpec and RoughSpec will, because of this, fail to utilize this function if the execution is terminated with an error.
In addition to what was mentioned above, data types and constructors can be marked as unsafe.
They are handled by terminating the process, and mentioning that they contain unsafe values.
