\vspace{-0.8cm}
\begin{figure}[H]
  \centering
  \begin{tikzpicture}[node distance=1.5 and 4.6, font=\footnotesize, ->, >={Stealth[sep]}]
  \tikzstyle{module} = [rectangle, draw, rounded corners=11, inner sep=7, minimum height=23]
  \tikzstyle{desc} = [font=\scriptsize, inner sep=1]
  \tikzstyle{desco} = [desc, pos=0.5, align=center, fill=white]
  \tikzstyle{desca} = [desc, pos=0.48, above=0.4ex]

  \node[module]  (lm)  {Lean Macro};
  \node[draw, left =0.4 of lm] (us)  {User};
  \node[module, below =of lm]      (l2h) {Lean to Haskell transpiler};
  \node[module, right =of l2h]     (cg)  {Haskell Conjecture generator};
  \node[module, above =of cg]      (h2l) {Conjectures to Lean code};

  \draw              (us)  -- (lm);
  \draw[shorten >=7] (lm)  -- (cg)  node[desco] {Size parameter for term \\ generation and aliases \\ to use for definitions};
  \draw              (lm)  -- (l2h) node[desco] {Set of definitions};
  \draw              (l2h) -- (cg)  node[desca] {Definitions translated to Haskell};
  \draw              (cg)  -- (h2l) node[desco] {List of conjectures \\ in an internal format};
  \draw              (h2l) -- (lm)  node[desca] {\raisebox{0pt}[\height][0pt]{Lean code of conjectures}};
\end{tikzpicture}

  \caption{An overview of the information flow of Pisa.}\label{fig:overview}
\end{figure}
\vspace{-0.3cm}

Pisa is constructed by multiple distinct modules that will be explained in this chapter.
An overview of how they interact can be seen in \cref{fig:overview}.

A rundown of Pisa starts at the end user working with Lean as an ITP.
They interact with Pisa by providing a set of definitions to explore through a macro within Lean.
An optional size parameter can also be specified to control how large the generated terms may be when searching for conjectures.
The given definitions, and all their transitive dependencies, are resolved and serialized by the exporter (Example can be seen in \cref{sec:lean-interface:exporter}).
This information is then dispatched to an executable containing the remaining subsystems.
Conjectures then replace the call site in the ITP upon request from the user by inserting the output of this executable.

To produce conjectures the executable uses QuickSpec and RoughSpec.
This first requires that the serialized Lean definitions are transformed into Haskell code.
This is presented in \cref{sec:pisa-haskell:translation}.
The generated Haskell code can then be run in an interpreter.
This is utilized by QuickSpec and RoughSpec to generate a set of conjectures, that finally is transformed back into Lean syntax.

Since the differences between the languages are non-trivial multiple approaches could be taken to solve the problem.
Several are presented and discussed in \cref{sec:discussion:alternative-approaches}.
