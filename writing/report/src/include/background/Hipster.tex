\section{Hipster}\label{sec:hipster}
Hipster is a similar project to Pisa, as in it adds a theory exploration tool into the ITP known as Isabelle/HOL \autocite{Hipster}.
This is accomplished using the following main parts.
First, Isabelle/HOL has a built-in code generator that is able to translate definitions into Haskell.
This allows the use of the second part, which is QuickSpec.
QuickSpec allows Hipster to find conjectures on the definitions, and these conjectures are then translated in to the proof style of Isabelle/HOL.
These conjectures are then attempted to be automatically proven by tooling in Isabelle/HOL, and are then presented to the end user.

Comparing this to Pisa, the main difference to this thesis is the presence of a code generator to Haskell.
Since Lean does not have a code generator to Haskell, one has to be implemented to attempt the same approach that was taken by Hipster.

There is a difference in how functions that are partially defined are handled between Isabelle/HOL and Lean.
In Isabelle/HOL, partially defined functions will, if using an unspecified input, return an arbitrary value.
Meanwhile, in Haskell there is a specified behavior, which is crashing.
Therefore, Hipster had to implement behavior to handle this difference.
In Lean, you are unable to construct a partially defined function, without marking it as such.
One has to match on all the cases that can occur from a data type.
Further, if a function is marked as partial, it cannot be used in proofs.
Thus, dealing with partial functions is not an aspect that needs to be handled.

One major aspect of Hipster that is not attempted is that of discarding lemmas that can be solved by routine reasoning.
Pisa does not attempt to do this.
Every part of the rewrite needs to be represented in the context, and this would therefore be a hindrance to the users of Pisa.
